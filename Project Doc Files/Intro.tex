\section*{Introdução}

Este documento foi desenvolvido com o intuito de documentar todo o processo de contrução do sistema
Devlean. Além disso, tem por objetivo apontar funcionalidades e eventuais falhas pertinentes às
telas. Este documento específico diz respeito ao \textbf{MÓDULO DE VENDAS}. Todo o versionamento
da documentação será feito via ferramenta \textit{Git}, pelo \textit{GitHub}.\\

A construção a qual este documento se baseia é a classe ``Livro''. Assim, veremos que sua 
estutura é básica, sendo: Capa de apresentação, Introdução, Explicação do modelo, Sumário e Capítulos. 
Veja mais detalhes sobre essas divisões na seção seguinte (Explicação do modelo).\\

Por fim, a documentação também tem por objetivo apontar para a equipe de desenvolvimento falhas no sistema 
e/ou no código. Se forem encontrados problemas no sistema, mudanças e melhorias serão necessárias. Estas
estarão indicadas no documento intitulado ``\textbf{REQUERIMENTO DE MUDANÇAS E MELHORIAS}''. Caso for necessário
mudar o código sem alterar a aplicação, seja para aprimorar a lógica, deixa-lo menos verboso ou qualquer
coisa do tipo, as alterações necessárias estarão presentes no documento intitulado ``\textbf{REQUERIMENTO
DE REFATORAÇÃO}''. Mais detalhes também estarão presentes na seção a seguir. Os padrões destes documentos
estão nas Figuras \ref{fig:modeloRequerimentoMudanca}, \ref{fig:checkedBoxModeloRequerimento} e
\ref{fig:modeloRequerimentoRefatorecao}.







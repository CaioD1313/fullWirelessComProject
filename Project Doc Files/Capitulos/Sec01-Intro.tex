\newpage
%%%%%%%%%%%%%%%%%%%%%%%%%%%%%%%%%%%%
%%%%%%%%     Introduction   %%%%%%%%
%%%%%%%%%%%%%%%%%%%%%%%%%%%%%%%%%%%%

\section{Introduction}\label{01Sec:Introduction}

This project aimed on the full development of a wireless communicating capable device,
from scratch all the way up to the final design. Throughout the development process, careful consideration
was given to hardware requirements and specifications. The firmware was optimized to ensure accurate temperature data
collection and efficient data transmission, while minimizing resource utilization and power consumption.
The firmware development encompassed various aspects, including the design of the firmware architecture,
algorithm implementation, and integration of communication protocols. \\ 

The following sections of this documentation provide detailed insights into the project's hardware and firmware
requirements, operation flow, device usage, limitations, and potential future improvements. The schematics and PCB layout
diagrams offer a comprehensive view of the hardware design, while the firmware section provides an in-depth analysis of the 
software implementation.


\subsection{Hardware Requirements}\label{01Sub:HardwareRequirements}

The chosen wireless protocol for this project was WiFi, and so, the ESP8266EX microcontroller was selected for its robustness, 
affordability, widespread acceptance in the firmware market and strong community support. Additionally, this MCU has the capability 
to work with RF signals at power levels of up to +20dBm in its TX, and sensitivities of -91dBm and -75dBm for 11Mbps and 54Mbps on 
its RX, respectively. Based on these values, under ideal conditions, signals can travel up to 300m. However, it is important to 
validate the project concept in real-world scenarios, where noisy environments and obstacles are almost always present. \\ 

For the temperature sensor, the TMP35 was chosen due to its low power consumption, ability to be powered off when not in use, 
good accuracy, and similarity to other widely used components in the market, such as the LM35. \\ 

As for the flash EEPROM memories, the MX25R3235F (32Mbit) and AT25SF081 (8Mbit) chips were selected for program and 
collected data storage, respectively. These ICs were chosen because they meet the size requirements (minimum 2MB for program and 
500kB for data) and operate under the serial SPI communication protocol.


\subsection{Firmware Requirements}\label{01Sub:FirmwareRequirements}

For the firmware, the development was chosen to be carried out using the Arduino IDE, as it works with C++,
a language accepted by the chosen microcontroller. Additionally, this IDE provides pre-developed board configurations,
which can be helpful during the prototyping phase. Moreover, it makes firmware uploading easier since we can utilize ready-made
modules for these tasks, such as the well-known ESP8266 NodeMCU 1.0.\\ 

For the WiFi protocol, the "ESP8266WiFi.h" library is used, licensed under the LGPL (Lesser General Public License) Version 2.1.
In broad terms, the license can be interpreted as follows: \\ 

\begin{flushright}
    \begin{quote}
        \textit{``Primarily used for software libraries, the GNU LGPL requires that derived works be licensed under the same 
        license,but works that only link to it do not fall under this restriction.''}
\end{quote}
\end{flushright}


The candidate chose to use this library because they do not have solid knowledge of the WiFi protocol (from the physical layer to 
the application layer) to develop their own library or subroutines. A full copy of the LGPL Ver 2.1 can be found at \cite{LGPL}. \\



For the communication between the microcontroller and the Flash Data Memory, the SPI communication protocol is used.
Also, the  it was implemented an algorithm for a linear time-invariant system, namely the Moving Average Filter. Such 
algorithm is required to ensure that the measured temperature remains in a stable state when it is saved in the flash data memory 
integrated into the hardware.









\subsection{Operation Flow}\label{01Sub:OperationFlow}

The final device workflow is simple: it stabilizes the reading/collecting of temperature data,
stores it on the flash EEPROM, and then opens communication with a access point, access a 
localhost webserver url and transmits the collected data. Finally, it erases the last $512,000$
samples of temperature and starts the process again.


\subsection{Limitations and Future Improvements}\label{01Sub:LimitationsAndFutureImprovements}


To reduce the code complexity, it was assumed that the server the board communicates with via the Access Point (AP) is a 
"localhost," and therefore, no SSL-HTTPS security configurations were implemented. Additionally, no ACK verification is performed 
to ensure that the server has indeed received the transmitted data. Furthermore, no HTTP POST messages containing SQL commands 
for writing to a hypothetical database are implemented. It is assumed that this set of business rules would be on the server side. 
Also, to avoid further increasing the code length (which has already been extended due to an unnecessary feature - reading 
temperature with precision and storing it in memory), no routines were implemented to ensure that the data was successfully written 
to the PCB memory. Finally, the device does not have any interface; it is simply plug and play. If there were any specific 
requirements, such an interface could be added subsequently.



%%%%%%%%%%%     FIM      %%%%%%%%%%%  
%%%%%%%%     Introduction   %%%%%%%%
%%%%%%%%%%%%%%%%%%%%%%%%%%%%%%%%%%%%
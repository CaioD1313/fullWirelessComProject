\newpage
%%%%%%%%%%%%%%%%%%%%%%%%%%%%%%%%%%%%
%%%%%%%%      Firmware      %%%%%%%%
%%%%%%%%%%%%%%%%%%%%%%%%%%%%%%%%%%%%

\section{Firmware}\label{04Sec:Firmware}


\subsection{Firmware Description}\label{04Sub:FirmwareDescription}

The ultimate goal of this firmware is to establish a complete wireless communication 
between a data collecting station (ESP8266EX) and an access point located 100m away 
using the WiFi protocol. The hardware collects data on-site, stores it with a size of 500kB, 
and then sends it to the access point. \\ 

For the purpose of data collection, a temperature measuring circuit was implemented
on the hardware using a TMP36 IC. A linear time-invariant system, namely the Moving 
Average Filter, is required to ensure that the measured temperature remains in a stable
state when it is saved in the flash data memory integrated into the hardware. \\ 

For the WiFi protocol, the library "ESP8266WiFi.h" was used. \\
  

For the communication between the microcontroller and the Flash Data Memory, it is used the SPI 
communication protocol.


\subsection{Firmware Architecture}\label{04Sub:FirmwareArchitecture}


\subsection{Features and Algorithms}\label{04Sub:FeaturesAndAlgorithms}


\subsection{Configurations and Parameters}\label{04Sub: ConfigurationsAndParameters}


\subsection{Interfaces and Protocols}\label{04Sub: InterfacesAndProtocols}




\begin{lstlisting}[language=Arduino]
    #include <iostream>

    #include <Arduino.h>

    void setup() {
        Serial.begin(9600);
        pinMode(LED_BUILTIN, OUTPUT);
    }
    
    void loop() {
        digitalWrite(LED_BUILTIN, HIGH);
        delay(500);
        digitalWrite(LED_BUILTIN, LOW);
        delay(500);
    }
    \end{lstlisting}







%%%%%%%%%%%%%    END   %%%%%%%%%%%%%
%%%%%%%%      Firmware      %%%%%%%%
%%%%%%%%%%%%%%%%%%%%%%%%%%%%%%%%%%%%
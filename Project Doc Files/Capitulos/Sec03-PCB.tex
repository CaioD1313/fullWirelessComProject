\newpage
%%%%%%%%%%%%%%%%%%%%%%%%%%%%%%%%%%%%
%%%%%%%%        PCB         %%%%%%%%
%%%%%%%%%%%%%%%%%%%%%%%%%%%%%%%%%%%%

\section{Printed Circuit Board}\label{03Sec:PCB}



    


\subsection{PCB Design}\label{03Sub:PCBDesign}


\subsection{PCB Specifications}\label{03Sub:PCBSpecifications}



\subsubsection{ESP8266EX Power Supply Design}\label{02SubSub:PowerSupplyDesign}


Prior to the power traces reaching the analog power-supply pins (Pin1, 3, 4, 28, 29), the inclusion of a 10 $\mu$F capacitor, working 
in conjunction with a 0.1 $\mu$F capacitor, becomes necessary. Additionally, an arrangement of a C circuit and an L circuit is advised 
for the power supplies of Pin3 and Pin4.The C-L-C circuit should be positioned as proximate as possible to the analog power-supply pin \cite{ESP8266HGL}. 
For more information about these components, refer to Figure \ref{02fig:analogAndDigitalSupplies1}.
It is crucial to acknowledge that the power supply pathway can be susceptible to damage as a consequence of abrupt current surges 
during the transmission of analog signals by ESP8266EX. Consequently, it is imperative to incorporate an additional 10$\mu$F capacitor, 
packaged in either 0603 or 0805 dimensions, to complement the existing 0.1$\mu$F capacitor \cite{ESP8266HGL}.

\subsubsection{Layers Design}\label{02SubSub:LayersDesign}

On \cite{ESP8266HGL} there are two suggested models regarding the number and dispositions of layers on a PCB
that uses ESP8266EX (and by extension, an 2.4GHz RF antenna). The model with four layers was chosen. Here are 
the specifications: 


\begin{itemize}
    \item The initial layer corresponds to the uppermost part and is designated as the TOP layer, utilized for 
    signal lines and component placement.
    \item The subsequent layer represents the GND layer, dedicated solely to establishing a comprehensive ground plane without 
    the inclusion of signal lines.
    \item The following layer constitutes the POWER layer, exclusively reserved for the positioning of power lines. In certain 
    unavoidable situations, it is admissible to incorporate some signal lines within this layer.
    \item Lastly, the final layer denotes the BOTTOM layer, exclusively allocated for signal line placement. It is discouraged 
    to mount components on this particular layer.
\end{itemize}

Also, the power tracks should have a minimum size of 15 mils. The value used was 20 mils.

\subsubsection{Reset Track Design}\label{02SubSub:ResetTrackDesign}

The EXT\_RSTB (Pin32) possesses an internal pullup resistor and operates on an active low mechanism. To mitigate the possibility
of external disturbances causing unwarranted resets, it is advised to maintain a concise PCB trace for EXT\_RSTB and 
incorporate an RC circuit at this pin \cite{ESP8266HGL} (see section \ref{02SubSub:RCDelayResetCircuitry}).




\subsection{Trace Routing}\label{03Sub:TraceRouting}




\subsection{Ground Planes}\label{03Sub:Ground planes}


\subsection{Tracks and Vias}\label{03Sub:HolesAndVias}



\subsection{RF Antenna}\label{03Sub:RFAntenna}

\cite{ESP8266HGL}
\ref{02fig:RFAntennaCircuitry}




%%%%%%%%%%%     FIM      %%%%%%%%%%%  
%%%%%%%%        PCB         %%%%%%%%
%%%%%%%%%%%%%%%%%%%%%%%%%%%%%%%%%%%%
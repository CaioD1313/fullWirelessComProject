%\documentclass{article}
%\documentclass[11pt,openany,brazil]{report}
\documentclass[12pt,brazil]{article}


%%%%%%%%%%%%%%%%%%%%%%%%%%%%%%%%%%%%%%%
%%%%%%% Pacotes de formatação: %%%%%%%%
%%%%%%%%%%%%%%%%%%%%%%%%%%%%%%%%%%%%%%%

\usepackage[utf8]{inputenc} % indica o uso dos caracteres da nossa língua
\usepackage[english]{babel} % indica que o texto está em português, assim atributos
                               % automáticos, tais como criação de capítulos, inserção de
                               % elementos no sumário e etc já são escritos em pt-br

\usepackage{ragged2e} % fornece o comando \justifying


\usepackage{geometry}
 \geometry{
 a4paper,
 total={170mm,257mm},
 left=20mm,
 top=40mm,
 bottom = 30mm,
 }
 

% Pacote para formatção e utilização de tabelas
\usepackage{tabularx}
\usepackage{multirow} % permite a mescla de colunas e criação de merges 

\usepackage{array} % pacote para manter tabela com wdith fixa



%Pacote para formatar as legendas das figuras, tabelas e etc
\usepackage[font=scriptsize,labelfont=bf]{caption}
\usepackage[font=scriptsize,labelfont=bf]{subcaption}


%pacote para aparecer subsubsection enumeradas, no corpo do texto, mas não no sumário
\setcounter{secnumdepth}{3}

%%%%%%%%%%%%%%%% FIM %%%%%%%%%%%%%%%%%%
%%%%%%% Pacotes de formatação: %%%%%%%%
%%%%%%%%%%%%%%%%%%%%%%%%%%%%%%%%%%%%%%%


%%% Pacote para usar begin landscape
\usepackage{pdflscape}



\usepackage{biblatex} %Imports biblatex package
\usepackage{csquotes}
\addbibresource{bibliography.bib}


%%%%%%%%%%%%%%%%%%%%%%%%%%%%%%%%%%%%%%%
%%%%%%%% Pacotes de figuras: %%%%%%%%%%
%%%%%%%%%%%%%%%%%%%%%%%%%%%%%%%%%%%%%%%
\usepackage{float} %força a figura a ficar no lugar abaixo do texto com o comando H

\usepackage{graphicx}
%\graphicspath{ {./imagens/} }


%%%%%%%%%%%%%%%% FIM %%%%%%%%%%%%%%%%%%
%%%%%%%% Pacotes de figuras: %%%%%%%%%%
%%%%%%%%%%%%%%%%%%%%%%%%%%%%%%%%%%%%%%%










%%%%%%%%%%%%%%%%%%%%%%%%%%%%%%%%%%%%%%%
%%%%%%%% Pacotes de cores:  %%%%%%%%%%%
%%%%%%%%%%%%%%%%%%%%%%%%%%%%%%%%%%%%%%%

\usepackage{color}
\usepackage{xcolor} % Cores extras overleaf




%% pacote para formatação de códigos de programação em seu formato nativo
%% Ex: ```sql SELECT * FROM DUMMY_TABLE```
\usepackage{listings}


\usepackage{courier}

%%% Define Custom IDE Colors %%%
\definecolor{arduinoGreen}    {rgb} {0.17, 0.43, 0.01}
\definecolor{arduinoGrey}     {rgb} {0.47, 0.47, 0.33}
\definecolor{arduinoOrange}   {rgb} {0.8 , 0.4 , 0   }
\definecolor{arduinoBlue}     {rgb} {0.01, 0.61, 0.98}
\definecolor{arduinoDarkBlue} {rgb} {0.0 , 0.2 , 0.5 }


%%% Define Arduino Language %%%
\lstdefinelanguage{Arduino}{
  language=C++, % begin with default C++ settings 
%
%
  %%% Keyword Color Group 1 %%%  (called KEYWORD3 by arduino)
  keywordstyle=\color{arduinoGreen},   
  deletekeywords={  % remove all arduino keywords that might be in c++
                break, case, override, final, continue, default, do, else, for, 
                if, return, goto, switch, throw, try, while, setup, loop, export, 
                not, or, and, xor, include, define, elif, else, error, if, ifdef, 
                ifndef, pragma, warning,
                HIGH, LOW, INPUT, INPUT_PULLUP, OUTPUT, DEC, BIN, HEX, OCT, PI, 
                HALF_PI, TWO_PI, LSBFIRST, MSBFIRST, CHANGE, FALLING, RISING, 
                DEFAULT, EXTERNAL, INTERNAL, INTERNAL1V1, INTERNAL2V56, LED_BUILTIN, 
                LED_BUILTIN_RX, LED_BUILTIN_TX, DIGITAL_MESSAGE, FIRMATA_STRING, 
                ANALOG_MESSAGE, REPORT_DIGITAL, REPORT_ANALOG, SET_PIN_MODE, 
                SYSTEM_RESET, SYSEX_START, auto, int8_t, int16_t, int32_t, int64_t, 
                uint8_t, uint16_t, uint32_t, uint64_t, char16_t, char32_t, operator, 
                enum, delete, bool, boolean, byte, char, const, false, float, double, 
                null, NULL, int, long, new, private, protected, public, short, 
                signed, static, volatile, String, void, true, unsigned, word, array, 
                sizeof, dynamic_cast, typedef, const_cast, struct, static_cast, union, 
                friend, extern, class, reinterpret_cast, register, explicit, inline, 
                _Bool, complex, _Complex, _Imaginary, atomic_bool, atomic_char, 
                atomic_schar, atomic_uchar, atomic_short, atomic_ushort, atomic_int, 
                atomic_uint, atomic_long, atomic_ulong, atomic_llong, atomic_ullong, 
                virtual, PROGMEM,
                Serial, Serial1, Serial2, Serial3, SerialUSB, Keyboard, Mouse,
                abs, acos, asin, atan, atan2, ceil, constrain, cos, degrees, exp, 
                floor, log, map, max, min, radians, random, randomSeed, round, sin, 
                sq, sqrt, tan, pow, bitRead, bitWrite, bitSet, bitClear, bit, 
                highByte, lowByte, analogReference, analogRead, 
                analogReadResolution, analogWrite, analogWriteResolution, 
                attachInterrupt, detachInterrupt, digitalPinToInterrupt, delay, 
                delayMicroseconds, digitalWrite, digitalRead, interrupts, millis, 
                micros, noInterrupts, noTone, pinMode, pulseIn, pulseInLong, shiftIn, 
                shiftOut, tone, yield, Stream, begin, end, peek, read, print, 
                println, available, availableForWrite, flush, setTimeout, find, 
                findUntil, parseInt, parseFloat, readBytes, readBytesUntil, readString, 
                readStringUntil, trim, toUpperCase, toLowerCase, charAt, compareTo, 
                concat, endsWith, startsWith, equals, equalsIgnoreCase, getBytes, 
                indexOf, lastIndexOf, length, replace, setCharAt, substring, 
                toCharArray, toInt, press, release, releaseAll, accept, click, move, 
                isPressed, isAlphaNumeric, isAlpha, isAscii, isWhitespace, isControl, 
                isDigit, isGraph, isLowerCase, isPrintable, isPunct, isSpace, 
                isUpperCase, isHexadecimalDigit, 
                }, 
  morekeywords={   % add arduino structures to group 1
                break, case, override, final, continue, default, do, else, for, 
                if, return, goto, switch, throw, try, while, setup, loop, export, 
                not, or, and, xor, include, define, elif, else, error, if, ifdef, 
                ifndef, pragma, warning,
                }, 
% 
%
  %%% Keyword Color Group 2 %%%  (called LITERAL1 by arduino)
  keywordstyle=[2]\color{arduinoBlue},   
  keywords=[2]{   % add variables and dataTypes as 2nd group  
                HIGH, LOW, INPUT, INPUT_PULLUP, OUTPUT, DEC, BIN, HEX, OCT, PI, 
                HALF_PI, TWO_PI, LSBFIRST, MSBFIRST, CHANGE, FALLING, RISING, 
                DEFAULT, EXTERNAL, INTERNAL, INTERNAL1V1, INTERNAL2V56, LED_BUILTIN, 
                LED_BUILTIN_RX, LED_BUILTIN_TX, DIGITAL_MESSAGE, FIRMATA_STRING, 
                ANALOG_MESSAGE, REPORT_DIGITAL, REPORT_ANALOG, SET_PIN_MODE, 
                SYSTEM_RESET, SYSEX_START, auto, int8_t, int16_t, int32_t, int64_t, 
                uint8_t, uint16_t, uint32_t, uint64_t, char16_t, char32_t, operator, 
                enum, delete, bool, boolean, byte, char, const, false, float, double, 
                null, NULL, int, long, new, private, protected, public, short, 
                signed, static, volatile, String, void, true, unsigned, word, array, 
                sizeof, dynamic_cast, typedef, const_cast, struct, static_cast, union, 
                friend, extern, class, reinterpret_cast, register, explicit, inline, 
                _Bool, complex, _Complex, _Imaginary, atomic_bool, atomic_char, 
                atomic_schar, atomic_uchar, atomic_short, atomic_ushort, atomic_int, 
                atomic_uint, atomic_long, atomic_ulong, atomic_llong, atomic_ullong, 
                virtual, PROGMEM,
                },  
% 
%
  %%% Keyword Color Group 3 %%%  (called KEYWORD1 by arduino)
  keywordstyle=[3]\bfseries\color{arduinoOrange},
  keywords=[3]{  % add built-in functions as a 3rd group
                Serial, Serial1, Serial2, Serial3, SerialUSB, Keyboard, Mouse,
                },      
%
%
  %%% Keyword Color Group 4 %%%  (called KEYWORD2 by arduino)
  keywordstyle=[4]\color{arduinoOrange},
  keywords=[4]{  % add more built-in functions as a 4th group
                abs, acos, asin, atan, atan2, ceil, constrain, cos, degrees, exp, 
                floor, log, map, max, min, radians, random, randomSeed, round, sin, 
                sq, sqrt, tan, pow, bitRead, bitWrite, bitSet, bitClear, bit, 
                highByte, lowByte, analogReference, analogRead, 
                analogReadResolution, analogWrite, analogWriteResolution, 
                attachInterrupt, detachInterrupt, digitalPinToInterrupt, delay, 
                delayMicroseconds, digitalWrite, digitalRead, interrupts, millis, 
                micros, noInterrupts, noTone, pinMode, pulseIn, pulseInLong, shiftIn, 
                shiftOut, tone, yield, Stream, begin, end, peek, read, print, 
                println, available, availableForWrite, flush, setTimeout, find, 
                findUntil, parseInt, parseFloat, readBytes, readBytesUntil, readString, 
                readStringUntil, trim, toUpperCase, toLowerCase, charAt, compareTo, 
                concat, endsWith, startsWith, equals, equalsIgnoreCase, getBytes, 
                indexOf, lastIndexOf, length, replace, setCharAt, substring, 
                toCharArray, toInt, press, release, releaseAll, accept, click, move, 
                isPressed, isAlphaNumeric, isAlpha, isAscii, isWhitespace, isControl, 
                isDigit, isGraph, isLowerCase, isPrintable, isPunct, isSpace, 
                isUpperCase, isHexadecimalDigit, 
                },      
%
%
  %%% Set Other Colors %%%
  stringstyle=\color{arduinoDarkBlue},    
  commentstyle=\color{arduinoGrey},    
%          
%   
  %%%% Line Numbering %%%%
  numbers=left,                    
  numbersep=5pt,                   
  numberstyle=\color{arduinoGrey},    
  %stepnumber=2,                      % show every 2 line numbers
%
%
  %%%% Code Box Style %%%%
  breaklines=true,                    % wordwrapping
  tabsize=2,         
  basicstyle=\ttfamily  
}

%%%% DEFINIÇÕES DE CORES HTML: NOSSO PADRÃO PARA UTILZIÇÃO NA DOCUMENTAÇÃO

% Padrão -> {NomeDaCor}{PadraoDaCor}{CódigoHexadecimal}

\definecolor{ClassColor}{HTML}{4EC9A7} %Cor das classes do C#
\definecolor{Control}{HTML}{FF6400} % Laranja da Devlean
\definecolor{SF}{HTML}{06428D} % Azul Smart Factory
\definecolor{TabelaBanco}{HTML}{315fbb} % Padrão de cor para tabelas no banco Devlean
\definecolor{IList}{HTML}{DCC774} %Cor das classes do C#
%%%%%%%%%%%%%%%% FIM %%%%%%%%%%%%%%%%%%
%%%%%%%% Pacotes de cores:  %%%%%%%%%%%
%%%%%%%%%%%%%%%%%%%%%%%%%%%%%%%%%%%%%%%






%%%%%%%%%%%%%%%%%%%%%%%%%%%%%%%%%%%%%%%
%%%%%%%% Pacotes para hyperlink %%%%%%%
%%%%%%%        no sumário       %%%%%%%
%%%%%%%%%%%%%%%%%%%%%%%%%%%%%%%%%%%%%%%


\usepackage{hyperref}
\hypersetup{
    colorlinks=true, %set true if you want colored links
    linktoc=all,     %set to all if you want both sections and subsections linked
    linkcolor=black,  %choose some color if you want links to stand out
    filecolor=blue,      
    urlcolor=blue,
    citecolor = black
}


%%%%%%%%%%%%%%%% FIM %%%%%%%%%%%%%%%%%%
%%%%%%%% Pacotes para hyperlink %%%%%%%
%%%%%%%        no sumário       %%%%%%%
%%%%%%%%%%%%%%%%%%%%%%%%%%%%%%%%%%%%%%%






%%%%%%%%%%%%%%%%%%%%%%%%%%%%%%%%%%%%%%%
%%%%%%%% Pacotes para BackGroud %%%%%%%
%%%%%%%        Template         %%%%%%%
%%%%%%%%%%%%%%%%%%%%%%%%%%%%%%%%%%%%%%%



%%%% Se quiser colocar apenas em algumas páginas, utilizar o campo \usepackage[pages=some] 'some' <-
%% Comandos para colocar background
% \BgThispage %% página que contém o background

% \clearpage 
% \usepackage[pages=all]{background}

% \backgroundsetup{
% scale=1,
% color=black,
% opacity=1,
% angle=0,
% contents={%
%   \includegraphics[width=\paperwidth,height=\paperheight]{imagens/templatePlataformaInovacao.png}
%   }%
% }


%%%%%%%%%%%%%%%% FIM %%%%%%%%%%%%%%%%%%
%%%%%%%% Pacotes para BackGroud %%%%%%%
%%%%%%%        Template         %%%%%%%
%%%%%%%%%%%%%%%%%%%%%%%%%%%%%%%%%%%%%%%



% %%%%%%%%%%%%%%%%%%%%%%%%%%%%%%%%%%%%%%%
% %%%%%%%   Pacotes para Fonte    %%%%%%%
% %%%%%%%        Arial            %%%%%%%
% %%%%%%%%%%%%%%%%%%%%%%%%%%%%%%%%%%%%%%%

% \usepackage{uarial}
% \setmainfont{Arial}


% %%%%%%%%%%%%%%%% FIM %%%%%%%%%%%%%%%%%%
% %%%%%%%   Pacotes para Fonte    %%%%%%%
% %%%%%%%        Arial            %%%%%%%
% %%%%%%%%%%%%%%%%%%%%%%%%%%%%%%%%%%%%%%%


\renewcommand*\footnoterule{}


%%%%%%%%%%%%%%%%%%%%%%%%%%%%%%%%%%%%%%%
%%%%%%%%% INÍCIO DOCUMENTO  %%%%%%%%%%%
%%%%%%%%%%%%%%%%%%%%%%%%%%%%%%%%%%%%%%%
\renewcommand{\thefootnote}{\Roman{footnote}}
\begin{document}



%%%%%%%%%%%%%%%%%%%%%%%%%%%%%%%%%%%%%%%
%%%%%%%   Capa do Documento     %%%%%%%
%%%%%%%%%%%%%%%%%%%%%%%%%%%%%%%%%%%%%%%





\pagenumbering{gobble}  % Comando para tirar a numeração de uma página 
                        % Nesse caso -> Capa
% \begin{figure}[H] % este H garante que a figura apareça aqui, nesta ordem
%     \centering % comando para centralizar a figura
%     \includegraphics[scale=0.4]{imagens/MGMLogo1.png} %scale é a resolução da                                         %imagem, escala. 1 = 100%
% \end{figure}

\begin{figure}[H] % este H garante que a figura apareça aqui, nesta ordem
\centering % comando para centralizar a figura
\vspace{5cm}
% \includegraphics[scale=0.4]{imagens/dev1_logo.png} %scale é a resolução da                                         %imagem, escala. 1 = 100%
\end{figure} \vspace{1cm}

\begin{center}
    \Huge \textbf{Full Wireless Communication Device}\vspace{8cm}
\end{center} 

\begin{center}
    \raggedleft{\textbf{Project documentation.} }  \vspace{4cm}
\end{center} 

\begin{center}
    \raggedleft{\textbf{Caio Dutra} } 
\end{center} 

\begin{center}
    \raggedleft{\textbf{May 2023} } \\ 
\end{center} 





% \begin{center}
  
    % \raggedleft \textbf{\Huge{Chamada}} \textbf{\Huge{\textcolor{SF}{Smart Factory}}} \footnote{\raggedleft\textcolor{Red}{Colocar o num/data do Edital aqui}}
 
 




% %Data de lançamento do documento
% \vspace{2.93cm} \today %21 de Fevereiro de 2022


   
   
% \end{center}



%%%%%%%%%%%%%%%   FIM   %%%%%%%%%%%%%%%
%%%%%%%   Capa do Documento     %%%%%%%
%%%%%%%%%%%%%%%%%%%%%%%%%%%%%%%%%%%%%%%




%%%%%%%%%%%%%%%%%%%%%%%%%%%%%%%%%%%%%%
%%%%%% Sumário do Documento     %%%%%%
%%%%%%%%%%%%%%%%%%%%%%%%%%%%%%%%%%%%%%
\justifying
\newpage
\tableofcontents %cria sumário

\clearpage % Comando para tirar a numeração de uma página 
           % Nesse caso -> Sumário (Tabel of contents)

%%%%%%%%%%%%%%%%%%%%%%%%%%%%%%%%%%%%%%
%%%%%% Sumário do Documento     %%%%%%
%%%%%%%%%%%%%%%%%%%%%%%%%%%%%%%%%%%%%%

%%%%%%%%%%%%%%%%%%%%%%%%%%%
% Aqui começa a paginação %
%%%%%%%%%%%%%%%%%%%%%%%%%%%

%%%%%% Esta linha retira a numeração das páginas
\pagenumbering{gobble}

%%%%%% Estas duas linhas fazem a numeração das páginas
% \pagenumbering{arabic}
% \setcounter{page}{1}
%%%%%%


%%%%%%%%%%%%%%%%%%%%%%%%%%
% Aqui acaba a paginação %
%%%%%%%%%%%%%%%%%%%%%%%%%%


%%%%%%%%%%%%%%%%%%%%%%%%%%%%%%%
% Inicialização dos Capítulos %
%%%%%%%%%%%%%%%%%%%%%%%%%%%%%%%






%%%%%%%%%%%%%%%%%%%%%%%%%%%%%%%%%%%%%%%%
%%%   Sec 01   -   Introduction      %%%
%%%%%%%%%%%%%%%%%%%%%%%%%%%%%%%%%%%%%%%%

\newpage
%%%%%%%%%%%%%%%%%%%%%%%%%%%%%%%%%%%%
%%%%%%%%     Introduction   %%%%%%%%
%%%%%%%%%%%%%%%%%%%%%%%%%%%%%%%%%%%%

\section{Introduction}\label{01Sec:Introduction}

This project aimed on the full development of a wireless communicating capable device,
from scratch all the way up to the final design. Throughout the development process, careful consideration
was given to hardware requirements and specifications. The firmware was optimized to ensure accurate temperature data
collection and efficient data transmission, while minimizing resource utilization and power consumption.
The firmware development encompassed various aspects, including the design of the firmware architecture,
algorithm implementation, and integration of communication protocols. \\ 

The following sections of this documentation provide detailed insights into the project's hardware and firmware
requirements, operation flow, device usage, limitations, and potential future improvements. The schematics and PCB layout
diagrams offer a comprehensive view of the hardware design, while the firmware section provides an in-depth analysis of the 
software implementation.


\subsection{Hardware Requirements}\label{01Sub:HardwareRequirements}

The chosen wireless protocol for this project was WiFi, and so, the ESP8266EX microcontroller was selected for its robustness, 
affordability, widespread acceptance in the firmware market and strong community support. Additionally, this MCU has the capability 
to work with RF signals at power levels of up to +20dBm in its TX, and sensitivities of -91dBm and -75dBm for 11Mbps and 54Mbps on 
its RX, respectively. Based on these values, under ideal conditions, signals can travel up to 300m. However, it is important to 
validate the project concept in real-world scenarios, where noisy environments and obstacles are almost always present. \\ 

For the temperature sensor, the TMP35 was chosen due to its low power consumption, ability to be powered off when not in use, 
good accuracy, and similarity to other widely used components in the market, such as the LM35. \\ 

As for the flash EEPROM memories, the MX25R3235F (32Mbit) and AT25SF081 (8Mbit) chips were selected for program and 
collected data storage, respectively. These ICs were chosen because they meet the size requirements (minimum 2MB for program and 
500kB for data) and operate under the serial SPI communication protocol.


\subsection{Firmware Requirements}\label{01Sub:FirmwareRequirements}

For the firmware, the development was chosen to be carried out using the Arduino IDE, as it works with C++,
a language accepted by the chosen microcontroller. Additionally, this IDE provides pre-developed board configurations,
which can be helpful during the prototyping phase. Moreover, it makes firmware uploading easier since we can utilize ready-made
modules for these tasks, such as the well-known ESP8266 NodeMCU 1.0.\\ 

For the WiFi protocol, the "ESP8266WiFi.h" library is used, licensed under the LGPL (Lesser General Public License) Version 2.1.
In broad terms, the license can be interpreted as follows: \\ 

\begin{flushright}
    \begin{quote}
        \textit{``Primarily used for software libraries, the GNU LGPL requires that derived works be licensed under the same 
        license,but works that only link to it do not fall under this restriction.''}
\end{quote}
\end{flushright}


The candidate chose to use this library because they do not have solid knowledge of the WiFi protocol (from the physical layer to 
the application layer) to develop their own library or subroutines. A full copy of the LGPL Ver 2.1 can be found at \cite{LGPL}. \\



For the communication between the microcontroller and the Flash Data Memory, the SPI communication protocol is used.
Also, the  it was implemented an algorithm for a linear time-invariant system, namely the Moving Average Filter. Such 
algorithm is required to ensure that the measured temperature remains in a stable state when it is saved in the flash data memory 
integrated into the hardware.









\subsection{Operation Flow}\label{01Sub:OperationFlow}

The final device workflow is simple: it stabilizes the reading/collecting of temperature data,
stores it on the flash EEPROM, and then opens communication with a access point, access a 
localhost webserver url and transmits the collected data. Finally, it erases the last $512,000$
samples of temperature and starts the process again.


\subsection{Limitations and Future Improvements}\label{01Sub:LimitationsAndFutureImprovements}


To reduce the code complexity, it was assumed that the server the board communicates with via the Access Point (AP) is a 
"localhost," and therefore, no SSL-HTTPS security configurations were implemented. Additionally, no ACK verification is performed 
to ensure that the server has indeed received the transmitted data. Furthermore, no HTTP POST messages containing SQL commands 
for writing to a hypothetical database are implemented. It is assumed that this set of business rules would be on the server side. 
Also, to avoid further increasing the code length (which has already been extended due to an unnecessary feature - reading 
temperature with precision and storing it in memory), no routines were implemented to ensure that the data was successfully written 
to the PCB memory. Finally, the device does not have any interface; it is simply plug and play. If there were any specific 
requirements, such an interface could be added subsequently.



%%%%%%%%%%%     FIM      %%%%%%%%%%%  
%%%%%%%%     Introduction   %%%%%%%%
%%%%%%%%%%%%%%%%%%%%%%%%%%%%%%%%%%%%

%%%%%%%%%%%%%%%   FIM  %%%%%%%%%%%%%%%%%
%%%   Sec 01   -   Introduction      %%%
%%%%%%%%%%%%%%%%%%%%%%%%%%%%%%%%%%%%%%%%




%%%%%%%%%%%%%%%%%%%%%%%%%%%%%%%%%%%%%%%%%%%%%%%%%%%%%%%%%
%%%   Sec 02   -       Schematics                     %%%
%%%%%%%%%%%%%%%%%%%%%%%%%%%%%%%%%%%%%%%%%%%%%%%%%%%%%%%%%


%%%%%%%%%%%%%%%%%%%%%%%%%%%%%%%%%%%%%%%%%%%%%%
%%%%%%%%          Schematics          %%%%%%%%
%%%%%%%%%%%%%%%%%%%%%%%%%%%%%%%%%%%%%%%%%%%%%%



\section{Schematics}\label{02Sec:Schematics}




\subsection{Sub Schematics}\label{02Sub:}


% \begin{figure}[H]
%     \centering
%     \includegraphics[scale = 0.8]{imagens/03DesenvolvimentoSolucao/tabelaDesenvolvimento.png}
%     %\caption{\textbf{Tela de novo cadastro de Pedido de Vendas}.}
%     \label{03fig:tabelaDesenvolvimento}
% \end{figure}


%%%%%%%%%%%%%%%%%%%  END  %%%%%%%%%%%%%%%%%%%%
%%%%%%%%          Schematics          %%%%%%%%
%%%%%%%%%%%%%%%%%%%%%%%%%%%%%%%%%%%%%%%%%%%%%%

%%%%%%%%%%%%%%%           FIM          %%%%%%%%%%%%%%%%%%
%%%   Sec 02   -       Schematics                     %%%
%%%%%%%%%%%%%%%%%%%%%%%%%%%%%%%%%%%%%%%%%%%%%%%%%%%%%%%%%

%%%%%%%%%%%%%%%%%%%%%%%%%%%%%%%%%%%%%%%%%%%%%%%%%%%%%%%%%%%%%%
%%%   Sec 03   -            PCB                            %%%
%%%%%%%%%%%%%%%%%%%%%%%%%%%%%%%%%%%%%%%%%%%%%%%%%%%%%%%%%%%%%%

\newpage
%%%%%%%%%%%%%%%%%%%%%%%%%%%%%%%%%%%%
%%%%%%%%        PCB         %%%%%%%%
%%%%%%%%%%%%%%%%%%%%%%%%%%%%%%%%%%%%

\section{Printed Circuit Board}\label{03Sec:PCB}



    


\subsection{PCB Design}\label{03Sub:PCBDesign}


\subsection{PCB Specifications}\label{03Sub:PCBSpecifications}



\subsubsection{ESP8266EX Power Supply Design}\label{02SubSub:PowerSupplyDesign}


Prior to the power traces reaching the analog power-supply pins (Pin1, 3, 4, 28, 29), the inclusion of a 10 $\mu$F capacitor, working 
in conjunction with a 0.1 $\mu$F capacitor, becomes necessary. Additionally, an arrangement of a C circuit and an L circuit is advised 
for the power supplies of Pin3 and Pin4.The C-L-C circuit should be positioned as proximate as possible to the analog power-supply pin \cite{ESP8266HGL}. 
For more information about these components, refer to Figure \ref{02fig:analogAndDigitalSupplies1}.
It is crucial to acknowledge that the power supply pathway can be susceptible to damage as a consequence of abrupt current surges 
during the transmission of analog signals by ESP8266EX. Consequently, it is imperative to incorporate an additional 10$\mu$F capacitor, 
packaged in either 0603 or 0805 dimensions, to complement the existing 0.1$\mu$F capacitor \cite{ESP8266HGL}.

\subsubsection{Layers Design}\label{02SubSub:LayersDesign}

On \cite{ESP8266HGL} there are two suggested models regarding the number and dispositions of layers on a PCB
that uses ESP8266EX (and by extension, an 2.4GHz RF antenna). The model with four layers was chosen. Here are 
the specifications: 


\begin{itemize}
    \item The initial layer corresponds to the uppermost part and is designated as the TOP layer, utilized for 
    signal lines and component placement.
    \item The subsequent layer represents the GND layer, dedicated solely to establishing a comprehensive ground plane without 
    the inclusion of signal lines.
    \item The following layer constitutes the POWER layer, exclusively reserved for the positioning of power lines. In certain 
    unavoidable situations, it is admissible to incorporate some signal lines within this layer.
    \item Lastly, the final layer denotes the BOTTOM layer, exclusively allocated for signal line placement. It is discouraged 
    to mount components on this particular layer.
\end{itemize}

Also, the power tracks should have a minimum size of 15 mils. The value used was 20 mils.

\subsubsection{Reset Track Design}\label{02SubSub:ResetTrackDesign}

The EXT\_RSTB (Pin32) possesses an internal pullup resistor and operates on an active low mechanism. To mitigate the possibility
of external disturbances causing unwarranted resets, it is advised to maintain a concise PCB trace for EXT\_RSTB and 
incorporate an RC circuit at this pin \cite{ESP8266HGL} (see section \ref{02SubSub:RCDelayResetCircuitry}).




\subsection{Trace Routing}\label{03Sub:TraceRouting}




\subsection{Ground Planes}\label{03Sub:Ground planes}


\subsection{Tracks and Vias}\label{03Sub:HolesAndVias}



\subsection{RF Antenna}\label{03Sub:RFAntenna}

\cite{ESP8266HGL}
\ref{02fig:RFAntennaCircuitry}




%%%%%%%%%%%     FIM      %%%%%%%%%%%  
%%%%%%%%        PCB         %%%%%%%%
%%%%%%%%%%%%%%%%%%%%%%%%%%%%%%%%%%%%

%%%%%%%%%%%%%%%             FIM             %%%%%%%%%%%%%%%%%%
%%%   Sec 03   -            PCB                            %%%
%%%%%%%%%%%%%%%%%%%%%%%%%%%%%%%%%%%%%%%%%%%%%%%%%%%%%%%%%%%%%%




%%%%%%%%%%%%%%%%%%%%%%%%%%%%%%%%%%%%%%%%%%%%%%%%%%%%%%%%%
%%%   Sec 04   -        Firmware                      %%%
%%%%%%%%%%%%%%%%%%%%%%%%%%%%%%%%%%%%%%%%%%%%%%%%%%%%%%%%%

\newpage
%%%%%%%%%%%%%%%%%%%%%%%%%%%%%%%%%%%%
%%%%%%%%      Firmware      %%%%%%%%
%%%%%%%%%%%%%%%%%%%%%%%%%%%%%%%%%%%%

\section{Firmware}\label{04Sec:Firmware}

\begin{lstlisting}[language=Arduino]
    #include <iostream>

    #include <Arduino.h>

    void setup() {
        Serial.begin(9600);
        pinMode(LED_BUILTIN, OUTPUT);
    }
    
    void loop() {
        digitalWrite(LED_BUILTIN, HIGH);
        delay(500);
        digitalWrite(LED_BUILTIN, LOW);
        delay(500);
    }
    \end{lstlisting}


\subsection{Sub Firmware}\label{04Sub:}




%%%%%%%%%%%%%    END   %%%%%%%%%%%%%
%%%%%%%%      Firmware      %%%%%%%%
%%%%%%%%%%%%%%%%%%%%%%%%%%%%%%%%%%%%

%%%%%%%%%%%%%%%           FIM          %%%%%%%%%%%%%%%%%%
%%%   Sec 04   -        Firmware                      %%%
%%%%%%%%%%%%%%%%%%%%%%%%%%%%%%%%%%%%%%%%%%%%%%%%%%%%%%%%%


%%%%%%%%%%%%%%%%%%%%%%%%%%%%%%%
%  Finalização dos Capítulos  %
%%%%%%%%%%%%%%%%%%%%%%%%%%%%%%%


\newpage

%Comando para o package{natbib}
%\bibliography{bibliografia}

%Comando para o package{biblatex}
\cleardoublepage
\phantomsection
\addcontentsline{toc}{chapter}{References}
\printbibliography



\end{document}

%%%%%%%%%%%%%%%%%%%%%%%%%%%%%%%%%%%%%%%
%%%%%%%%%   FIM DOCUMENTO   %%%%%%%%%%%
%%%%%%%%%%%%%%%%%%%%%%%%%%%%%%%%%%%%%%%